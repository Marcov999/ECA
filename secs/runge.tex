\begin{lm} \label{taglio}
  Sia $K \subset \subset \mathbb{C}$ compatto, $V \supset K$ un intorno aperto. Allora esiste $g \in C^{\infty}(\mathbb{C})$ t.c. $g\restrict{K}=1$ e $supp(g) \subset V$ ($\implies g\restrict{\mathbb{C}\setminus V}\equiv 0$) [ricordiamo che $supp(g)=\overline{\{z \in \mathbb{C} \mid g(z)\not=0\}}$].
\end{lm}

\begin{proof}
  Sia $h: \mathbb{R} \longrightarrow \mathbb{R}$ data da $h(t)=\begin{cases}
    0 & \mbox{se }t\le 0\\ e^{-1/t} & \mbox{se }t>0
\end{cases}$, $h \in C^{\infty}(\mathbb{R})$. Sia $\eta: \mathbb{C} \longrightarrow \mathbb{C}$ data da $\eta(z)=\dfrac{h(1-|z|^2)}{h(1-|z|^2)+h(|z|^2-1/4)}$. $\eta \in C^{\infty}(\mathbb{C}), \eta(\mathbb{C})=[0, 1]$.
$\eta\restrict{D(0, 1/2)}\equiv 1$ e $\eta\restrict{\mathbb{C}\setminus \mathbb{D}} \equiv 0$. Dato $p \in K$, sia $r_p>0$ t.c. $D(p, 2r_p) \subset V$.
Allora, per compattezza di $K$, esistono $p_1, \dots, p_k \in K$ t.c. $\displaystyle K \subset \bigcup_{j=1}^k D(p_j, r_{p_j}/2) \subset \bigcup_{j=1}^k D(p_j, 2r_{p_j}) \subset V$. Poniamo $\displaystyle W=\bigcup_{j=1}^k D(p_j, r_{p_j})$.
Sia $g_j:\mathbb{C} \longrightarrow \mathbb{R}$, $g_j=\begin{cases}
  \eta\left(\dfrac{z-p_j}{r_{p_j}}\right) & \mbox{se }z\in D(p_j, 2r_{p_j})\\ 0 & \mbox{se }z\in\mathbb{C} \setminus \overline{D(p_j, r_{p_j})}
\end{cases}$, che è ben definita per come è definita $\eta$. $g_j \in C^{\infty}(\mathbb{C})$. Sia $g: \mathbb{C} \longrightarrow \mathbb{R}$, $\displaystyle g(z)=1-\prod_{j=1}^k (1-g_j(z))$. $g \in C^{\infty}(\mathbb{C})$.
Se $z \in K$, esiste $j$ t.c. $z \in D(p_j, r_{p_j}/2) \implies g_j(z)=1 \implies g(z)=1$. Se $z \not\in \overline{W}$, $z \not\in\overline{D(p_j, r_{p_j})}$ per ogni $j=1, \dots, k$ $\implies$ $g_j(z)=0$ per ogni $j$ $\implies$ $g(z)=0$ $\implies$ $supp(g) \subseteq \overline{W} \subset V$.
\end{proof}

\begin{thm}
  (Teorema di Cauchy generalizzato) Sia $\Omega \subset \subset \mathbb{C}$ un dominio limitato t.c. $\partial\Omega$ sia un numero finito di curve di Jordan. Sia $u \in C^1(\overline{\Omega}, \mathbb{C})$, cioè esiste $U$ intorno aperto di $\overline{\Omega}$ su cui $u$ si estende di classe $C^1$.
  Allora per ogni $w \in \Omega$ si ha $\displaystyle u(w)=\dfrac{1}{2\pi i}\int_{\partial \Omega} \dfrac{u(z)}{z-w}\diff z+\dfrac{1}{2\pi i}\int_{\Omega} \dfrac{\partial u/\partial\bar{z}}{z-w}\diff z \wedge \diff \bar{z}$.
\end{thm}

\begin{proof}
  Usando la formula di Gauss-Green abbiamo che $\displaystyle \int_{\partial \Omega}(f\diff x+g\diff y)=\int_{\Omega}\left(\dfrac{\partial g}{\partial x}-\dfrac{\partial f}{\partial y}\right)\diff x\diff y$. Sia $v \in C^1(\overline{\Omega}, \mathbb{C}), f=\mathfrak{Re}(v), g=\mathfrak{Im}(v)$.
  $v=f+ig, \diff z=\diff x+i\diff y \implies v\diff z=(f\diff x-g\diff y)+i(g\diff x+f\diff y)$.
  Per Gauss-Green, $\displaystyle \int_{\partial \Omega} v\diff z=\int_{\Omega} \left(-\dfrac{\partial g}{\partial x}-\dfrac{\partial f}{\partial y}\right)\diff x\diff y+i\int_{\Omega} \left(\dfrac{\partial f}{\partial x}-\dfrac{\partial g}{\partial y}\right)\diff x\diff y$.
  $\dfrac{\partial v}{\partial\bar{z}}=\dfrac{1}{2}\left(\dfrac{\partial v}{\partial x}+i\dfrac{\partial v}{\partial y}\right)=\dfrac{1}{2}\left(\dfrac{\partial f}{\partial x}-\dfrac{\partial g}{\partial y}\right)+\dfrac{1}{2}i\left(\dfrac{\partial f}{\partial y}+\dfrac{\partial g}{\partial x}\right)$.
  $\diff \bar{z}\wedge\diff z=(\diff x-i\diff y)\wedge(\diff x+i\diff y)=2i\diff x\wedge\diff y$.
  $\dfrac{\partial v}{\partial\bar{z}}\diff\bar{z}\wedge \diff z=\left[-\left(\dfrac{\partial f}{\partial y}+\dfrac{\partial g}{\partial x}\right)+i\left(\dfrac{\partial f}{\partial x}-\dfrac{\partial g}{\partial y}\right)\right]\diff x\wedge\diff y$.
  Allora $\displaystyle \int_{\partial \Omega}v\diff z=\int_{\Omega} \dfrac{\partial v}{\partial\bar{z}}\diff\bar{z}\wedge\diff z$ $(\star)$.
  Con il teorema di Stokes, $\displaystyle \int_{\partial \Omega}v\diff z=\int_{\Omega} \diff(v\diff z)=\int_{\Omega} \diff v\wedge\diff z=\int_{\Omega} \left(\dfrac{\partial v}{\partial z}\diff z+\dfrac{\partial v}{\partial\bar{z}}\diff\bar{z}\right)\wedge\diff z=\int_{\Omega} \dfrac{\partial v}{\partial\bar{z}} \diff \bar{z}\wedge\diff z$.
  Fissato $w \in \Omega$ poniamo $v(z)=\dfrac{u(z)}{z-w}$ su $\Omega_{\epsilon}=\Omega \cap \{|z-w|>\epsilon\}$ dove $\epsilon>0$ è sufficientemente piccolo. $\partial\Omega_{\epsilon}=\partial\Omega\cup\partial D(w, \epsilon)$. $v \in C^1(\overline{\Omega}_{\epsilon})$.
  Per $(\star)$, $\displaystyle \int_{\partial\Omega_{\epsilon}} \dfrac{u(z)}{z-w}\diff z=\int_{\Omega_{\epsilon}} \dfrac{\partial u/\partial\bar{z}}{z-w}\diff\bar{z}\wedge\diff z$.
  Parametrizziamo $\partial D(w, \epsilon)$ con $\gamma(t)=w+\epsilon e^{it}, t \in [0, 2\pi]$, abbiamo allora che
  $\displaystyle \int_{\partial\Omega_{\epsilon}} \dfrac{u(z)}{z-w}\diff z=\int_{\partial\Omega} \dfrac{u(z)}{z-w}\diff z-\int_{\partial D(w, \epsilon)} \dfrac{u(z)}{z-w}\diff z=\int_{\partial\Omega} \dfrac{u(z)}{z_{\epsilon}-w}\diff z-\int_0^{2\pi} u(w+\epsilon e^{it})i\diff t$.
  Mandando $\epsilon$ a $0$, dato che $u$ è continua, otteniamo $\displaystyle \int_{\partial\Omega} \dfrac{u(z)}{z-w}\diff z-2\pi iu(w)$, mentre $\displaystyle \int_{\Omega_{\epsilon}} \dfrac{\partial u/\partial\bar{z}}{z-w}\diff\bar{z}\wedge\diff z$ tende a $\displaystyle \int_{\Omega} \dfrac{\partial u/\partial\bar{z}}{z-w}\diff\bar{z}\wedge\diff z$ perché $\dfrac{1}{z-w}$ è integrabile sugli aperti limitati di $\mathbb{C}$. Mettendo tutto insieme si ha la tesi.
\end{proof}

\begin{defn}
  L'\textit{equazione di Cauchy-Riemann non omogenea} è $\dfrac{\partial u}{\partial\bar{z}}=\varphi$ dove l'incognita è $u$.
\end{defn}

\begin{lm} \label{misura}
  Sia $K \subset\subset \mathbb{C}$ compatto e $\mu$ una misura con $supp(\mu)=K$. Allora l'integrale $\displaystyle u(w)=\int \dfrac{1}{w-z}\diff\mu(z)$ definisce una funzione $u \in \mathcal{O}(\mathbb{C}\setminus K)$.
\end{lm}

\begin{proof}
  Una misura $\mu$ con $supp(\mu)=K$ è un elemento $\mu \in (C^0(K))^*$ continuo. $\displaystyle \int \dfrac{1}{w-z} \diff\mu(z)=\mu\left(\dfrac{1}{w-\cdot}\right), \dfrac{1}{w-\cdot} \in C^0(K)$ se $w\not\in K$.
  Sia $a \not\in K$, $r>0$ t.c. $\overline{D(a, r)}\cap K=\emptyset$, $w \in D(a, r)$.
  $\displaystyle \dfrac{1}{w-z}=\dfrac{1}{(a-z)\left(1-\dfrac{a-w}{a-z}\right)}=\sum_{n \ge 0} \dfrac{(a-w)^n}{(a-z)^{n+1}}$ in quanto $\left|\dfrac{a-w}{a-z}\right|<1$ per ogni $z \in K, w \in D(a, r)$, quindi $\displaystyle \mu\left(\dfrac{1}{w-z}\right)=\sum_{n\ge 0} (a-w)^n\mu\left(\dfrac{1}{(a-z)^{n+1}}\right)$ è una serie di potenze in $w$ che converge in $D(a,r)$ $\implies$ $u \in \mathcal{O}(\mathbb{C}\setminus K)$.
\end{proof}

\begin{thm}
  Sia $\varphi \in C^k(\mathbb{C})$ a supporto compatto ($\varphi \in C^k_{\text{\textsc{c}}}(\mathbb{C})$). Allora esiste $u \in C^k(\mathbb{C})$ t.c. $\dfrac{\partial u}{\partial\bar{z}}=\varphi$.
\end{thm}

\begin{proof}
  Poniamo $\displaystyle u(w)=\dfrac{1}{2\pi i}\int \dfrac{\varphi(z)}{w-z} \diff\bar{z}\wedge\diff z$. È la $u$ data dal lemma \ref{misura} con $\mu=(2\pi i)^{-1}\varphi(\diff\bar{z}\wedge\diff z)=-\dfrac{1}{\pi}\varphi\diff y\diff x$.
  $u \in \mathcal{O}(\mathbb{C}\setminus K)$. Facciamo un cambiamento di variabile: $\zeta=w-z \implies z=w-\zeta, \diff\zeta=-\diff z, \diff\bar{\zeta}=-\diff\bar{z}$.
  $u(w)=\dfrac{1}{2\pi i} \int_{\mathbb{C}} \dfrac{\varphi(w-\zeta)}{\zeta}\diff\bar{\zeta}\wedge\diff\zeta$. Siccome $1/\zeta$ è integrabile sui compatti di $\mathbb{C}$ e $\varphi \in C^k_{\text{\textsc{c}}}(\mathbb{C})$ possiamo derivare sotto il segno di integrale e le derivate sono continue, $u \in C^k(\mathbb{C})$.
  $\displaystyle \dfrac{\partial u}{\partial \bar{w}}(w)=\dfrac{1}{2\pi i}\int_{\mathbb{C}} \dfrac{\frac{\partial \varphi}{\partial \bar{z}}(w-\zeta)}{\zeta} \diff \bar{z}\wedge\diff z=-\dfrac{1}{2\pi i}\int_{\mathbb{C}} \dfrac{\frac{\partial\varphi}{\partial\bar{z}}(z)}{z-w}\diff\bar{z}\wedge\diff z=\dfrac{1}{2\pi i}\int_{\mathbb{C}} \dfrac{\frac{\partial\varphi}{\partial\bar{z}}(z)}{z-w}\diff z\wedge\diff\bar{z}=\dfrac{1}{2\pi i}\int_{\Omega} \dfrac{\frac{\partial\varphi}{\partial\bar{z}}(z)}{z-w}\diff z\wedge\diff\bar{z}$ dove $\Omega \subset \subset \mathbb{C}$ è un disco con $K \subset\subset \Omega$.
  Quindi $\varphi\restrict{\partial \Omega}\equiv 0$ e il teorema di Cauchy generalizzato ci dà $\dfrac{\partial u}{\partial\bar{w}}(w)=\varphi(w)$ per ogni $w \in K$.
\end{proof}

\begin{oss}
  Se $K=supp(\varphi)$, $u \in \mathcal{O}(\mathbb{C} \setminus K)$.
\end{oss}

\begin{oss}
  Non è detto che $u$ abbia supporto compatto.
\end{oss}

\begin{oss}
  $u$ è unica a meno di funzioni in $\mathcal{O}(\mathbb{C})$.
\end{oss}

\begin{defn}
  Ricordiamo che se $K \subset\subset \mathbb{C}$ è compatto e $f \in C^0(K)$, allora poniamo $\|f\|_K=\sup_{z \in K}|f(z)|$.
  Definiamo adesso $\mathcal{O}(K)=\{f \in C^0(K) \mid$ esiste $(U, \tilde{f})$ dove $U \supset K$ è un intorno aperto di $K$, $\tilde{f} \in \mathcal{O}(U)$ e $\tilde{f}\restrict{K}\equiv f\}$.
\end{defn}

\begin{ex}
  Se $w \not\in K, f(z)=\dfrac{1}{w-z}$, allora $f \in \mathcal{O}(K)$.
\end{ex}

\begin{thm}
  (Primo teorema di Runge) Sia $K \subset\subset \mathbb{C}$ compatto, $\Omega \subseteq \mathbb{C}$ un intorno aperto di $K$. Le seguenti sono equivalenti:
  \begin{nlist}
    \item ogni $f \in \mathcal{O}(K)$ può essere approssimata uniformemente su $K$ da funzioni in $\mathcal{O}(\Omega)$;
    \item $\Omega \setminus K$ non ha componenti connesse relativamente compatte in $\Omega$;
    \item per ogni $z \in \Omega \setminus K$ esiste $f \in \mathcal{O}(\Omega)$ t.c. $|f(z)|>\|f\|_K$.
  \end{nlist}
\end{thm}

\begin{proof}
  (iii) $\implies$ (ii) Se (ii) è falso, esiste $U$ componente connessa di $\Omega \setminus K$ con $\overline{U} \subset \Omega$ e $\partial U \subseteq K$, dunque per il principio del massimo abbiamo, per ogni $g \in \mathcal{O}(\Omega)$ e per ogni $z \in U$, che $|g(z)|\le \max_{\zeta \in \partial U} |g(\zeta)| \le \|g\|_K$, contro (iii).

  (i) $\implies$ (ii) Se (ii) è falso, esiste $U$ componente connessa di $\Omega \setminus K$ con $\overline{U} \subset \Omega$ e $\partial U \subseteq K$. Sia $w \in U$, $f(z)=\dfrac{1}{w-z}$ e $f \in \mathcal{O}(K)$.
  Se (i) fosse vera esisterebbe $\{f_n\} \in \mathcal{O}(\Omega)$ t.c. $\|f_n-f\|_K \longrightarrow 0$ $\implies$ $\|f_m-f_n\|_K \longrightarrow 0$ per $m, n \longrightarrow +\infty$.
  Sempre per il principio del massimo, per ogni $z \in U$ $|g(z)|\le \max_{\zeta \in \partial U} |g(\zeta)| \le \|g\|_K$ $\implies$ $\|f_m-f_n\|_{\overline{U}} \longrightarrow 0$ per $m,n \longrightarrow +\infty$ $\implies$ $\{f_n\}$ è di Cauchy in $C^0(U \cup K)$ $\implies$ converga a una $F \in C^0(U \cup K) \subset C^0(\overline{U})$.
  Per il teorema di Weierestrass, $F \in \mathcal{O}(U)$. Su $K$ abbiamo che $(w-z)F(z)\equiv 1$. Ma allora applicando il principio del massimo a $(w-z)F(z)-1 \in \mathcal{O}(U) \cap C^0(\overline{U})$ otteniamo $(w-z)F(z)-1 \equiv 0$ su tutto $U$, impossibile (in $w$ fa $-1$).

  (ii) $\implies$ (i) Sia $\mathcal{O}(\Omega)\restrict{K}=\{f\restrict{K} \mid f \in \mathcal{O}(\Omega)\}$. Vogliamo $\mathcal{O}(\Omega)\restrict{K}$ denso in $\mathcal{O}(K)$ rispetto a $\|\cdot\|_K$.
  Per il teorema di Hahn Banach, se $W$ è un sottospazio vettoriale di $V$ con $W, V$ spazi vettoriali topologici, allora $W$ è denso in $V$ $\iff$ l'unico elemento di $V^*$ che si annulla su $W$ è $0$ $\iff$ per ogni $\mu \in V^*$ t.c. $\mu\restrict{W}\equiv 0$, $\mu \equiv 0$.
  Per avere la tesi serve quanto segue: se $\mu$ è una misura su $K$ (cioè un elemento del duale di $C^0(K)$) t.c. per ogni $f \in \mathcal{O}(\Omega)\restrict{K}$ $\displaystyle \int f(z)\diff\mu(z)=0$ $(\star)$ allora per ogni $g \in \mathcal{O}(K)$ $\displaystyle \int g(z)\diff\mu(z)=0$.
  Dato $\mu$ che soddisfa $(\star)$ definiamo $\varphi:\mathbb{C}\setminus K \longrightarrow \mathbb{C}$ ponendo $\displaystyle \varphi(w)=\int_{\mathbb{C}} \frac{1}{z-w}\diff\mu(z)=\int_K \frac{1}{z-w}\diff\mu(z)$.
  Abbiamo visto che, se $\varphi \in \mathcal{O}(\mathbb{C}\setminus K)$ e $w\not\in\Omega$, $(z-w)^{-1} \in \mathcal{O}(\Omega)$ e per $(\star)$ $\varphi\restrict{\mathbb{C}\setminus \Omega}\equiv 0$ $\implies$ $\varphi \equiv 0$ su ogni componente connessa di $\mathbb{C}\setminus K$ che interseca $\mathbb{C}\setminus\Omega$.
  Per ogni $n \ge 0$ $z^n \in \mathcal{O}(\Omega)$, dunque per $(\star)$ $\displaystyle \int z^n \diff\mu(z)=0$. Ma $(z-w)^{-1}$ si può sviluppare in una serie di potenze in $z$ che converge uniformemente su $K$ non appena $|w|>\|z\|_K$ $\implies$ $\varphi \equiv 0$ sulla componente connessa illimitata di $\mathbb{C}\setminus K$.
  Mancano solo le componenti connesse limitate di $\mathbb{C}\setminus K$ con chiusura contenuta in $\Omega$, cioè le componenti connesse di $\Omega\setminus K$ relativamente compatte in $\Omega$. Ma per (ii) non ce ne sono $\implies$ $\varphi\restrict{\mathbb{C}\setminus K}\equiv 0$. Sia $g \in \mathcal{O}(K)$, e sia $U \supset K$ un intorno aperto t.c. $g \in \mathcal{O}(U)$ con $\partial U$ "buono" (per fare l'integrale).
  Sia $\psi \in C^{\infty}(\mathbb{C})$ t.c. $\psi\restrict{K}\equiv 1$ e $supp(\psi) \subset \subset U$ data dal lemma \ref{taglio}.
  Abbiamo, per il teorema di Cauchy generalizzato, per ogni $w \in K$ $\displaystyle g(w)=\psi(w)g(w)=\frac{1}{2\pi i}\int_U \frac{g(z)}{z-w}\frac{\partial \psi}{\partial \bar{z}}(z)\diff z\wedge\diff\bar{z}+\frac{1}{2\pi i}\int_{\partial U} \frac{g(z)\psi(z)}{z-w}\diff z$.
  $\psi \restrict{\partial U}\equiv 0$, dunque il secondo integrale è nullo. Inoltre, $\psi\restrict{K}\equiv 1 \implies \dfrac{\partial \psi}{\partial \bar{z}}\restrict{K}\equiv 0$, quindi possiamo integrare su $U \setminus K$.
  Il risultato è dunque uguale a $\displaystyle \frac{1}{2\pi i}\int_{U \setminus K} \frac{g(z)}{z-w}\frac{\partial \psi}{\partial\bar{z}}(z)\diff z\wedge\diff\bar{z}$.
  Allora $\displaystyle \int g(w)\diff\mu(z)=\frac{1}{2\pi i}\int \diff\mu(z)\int_{U\setminus K} \frac{g(z)}{z-w}\frac{\partial \psi}{\partial\bar{z}}(z)\diff z\wedge\diff\bar{z}=-\frac{1}{2\pi i}\int_{U\setminus K} g(z)\frac{\partial \psi}{\partial\bar{z}}(z)\left(\int \frac{1}{w-z} \diff\mu(z)\right)\diff z\wedge\diff\bar{z}=0$ perché $\displaystyle \int \frac{1}{w-z} \diff\mu(z)=\varphi(z)$ e $\varphi\restrict{U \setminus K}\equiv 0$.

  (i)+(ii) $\implies$ (iii) Fissiamo $z_0 \in \Omega \setminus K$. Sia $D \subset \Omega \setminus K$ un disco chiuso di centro $z_0$. Le componenti connesse di $\Omega \setminus (K \cup D)$ sono lo stesse di $\Omega \setminus K$ con una a cui è stato tolto $D$. In particolare $K \cup D$ soddisfa (ii).
  La funzione $g$ che è $0$ su $K$ e $1$ su $D$ appartiene a $\mathcal{O}(K \cup D)$, dunque per (i) può essere approssimata da funzioni in $\mathcal{O}(\Omega)$ $\implies$ esiste $f \in \mathcal{O}(\Omega)$ t.c. $\|f\|_K<1/2$ e $\|f-1\|_D<1/2$ $\implies$ $|f(z_0)|>1-1/2=1/2$ $\implies$ $\|f\|_K<1/2<|f(z_0)|$.
\end{proof}

\begin{oss}
  Se $U \subset \Omega$ è aperto relativamente compatto in $\Omega$ ($\implies \partial U\subset \Omega$) allora per ogni $f \in \mathcal{O}(\Omega)$ $\|f\|_{\overline{U}}=\|f\|_{\partial U}$ per il principio del massimo.
\end{oss}

\begin{defn}
  Sia $\Omega \subseteq \mathbb{C}$ aperto, $K\subset\subset\Omega$ compatto. L'\textsc{inviluppo olomorfo di $K$ in $\Omega$} è $\widehat{K}_{\Omega}=\{z \in \Omega \mid |f(z)|\le\|f\|_K\forall f \in \mathcal{O}(\Omega)\}$.
\end{defn}

\begin{prop} \label{inviluppo}
  Sia $\Omega \subseteq \mathbb{C}$ aperto, $K\subset\subset\Omega$ compatto. Allora:
  \begin{nlist}
    \item per ogni $f \in \mathcal{O}(\Omega)$ $\|f\|_{\widehat{K}_{\Omega}}=\|f\|_K$;
    \item $K \subset \widehat{K}_{\Omega}$ e $\widehat{(\widehat{K}_\Omega)}_{\Omega}=\widehat{K}_{\Omega}$;
    \item $\widehat{K}_{\Omega}=K \iff \mathcal{O}(\Omega)\restrict{K}$ è denso in $\mathcal{O}(K)$;
    \item $d(w, \widehat{K}_{\Omega})=d(w, K)$ per ogni $w \in \mathbb{C}\setminus\Omega$. In particolare $d(\widehat{K}_{\Omega}, \mathbb{C}\setminus\Omega)=d(K, \mathbb{C}\setminus\Omega)$;
    \item $\widehat{K}_{\Omega}$ è compatto;
    \item $\widehat{K}_{\Omega}$ è l'unione di $K$ e delle componenti connesse di $\Omega \setminus K$ relativamente compatte in $\Omega$;
    \item $\mathbb{C}\setminus \widehat{K}_{\Omega}$ ha solo un numero finito di componenti connesse, nessuna contenuta in $\Omega$.
  \end{nlist}
\end{prop}

\begin{proof}
  \begin{nlist}
    \item Ovvia.
    \item Ovvia.
    \item Viene dal primo teorema di Runge.
    \item $w \not\in \Omega \implies (z-w)^{-1} \in \mathcal{O}(\Omega) \implies$ per ogni $z \in \widehat{K}_{\Omega}$ $\displaystyle \frac{1}{|z-w|}\le \sup_{\zeta \in K} \frac{1}{|\zeta-w|} \implies |z-w| \ge \inf_{\zeta \in K} |\zeta-w|=d(w, K)$.
    $\displaystyle d(w, \widehat{K}_{\Omega})=\inf_{z \in \widehat{K}_{\Omega}}|z-w| \ge d(w, K)$. La disuguaglianza opposta segue da $K \subseteq \widehat{K}_{\Omega}$.
    \item Usando $f(z)=z$ otteniamo che $\widehat{K}_{\Omega}$ è limitato. Il punto (iv) ci assicura che $\overline{\widehat{K}_{\Omega}} \subset \Omega$ e infine $\widehat{K}_{\Omega}$ è chiuso (segue dalla definizione).
    \item Sia $U$ una componente connessa di $\Omega \setminus K$ relativamente compatta in $\Omega$ $\implies$ $\partial U \subseteq K$, dunque per l'osservazione precedente per ogni $f \in \mathcal{O}(\Omega)$ $\|f\|_U \le \|f\|_K \implies U \subset \widehat{K}_{\Omega}$.
    Sia $K_1=K \cup$ le componenti connesse di $\Omega \setminus K$ relativamente compatte in $\Omega$. Abbiamo $K_1 \subseteq \widehat{K}_{\Omega}$ inoltre $K_1$ è chiuso (in quanto è contenuto in un compatto contenuto in $\Omega$ e $\Omega \setminus K_1$ è l'unione delle rimanenti componenti connesse di $\Omega$, che essendo aperto a componenti connesse aperte). Quindi $K_1$ è compatto e nessuna componente connessa di $\Omega \setminus K_1$ è relativamente compatta in $\Omega$. Per il primo teorema di Runge, $K_1=\widehat{(K_1)}_{\Omega}$.
    $K \subseteq K_1 \implies \widehat{K}_{\Omega} \subseteq \widehat{(K_1)}_{\Omega} \implies K_1=\widehat{K}_{\Omega}$.
    \item $\widehat{K}_{\Omega}$ è compatto. Quindi $\mathbb{C} \setminus \widehat{K}_{\Omega}$ ha una sola componente connessa illimitata $U_0$ che contiene il complementare di un disco contenente $\widehat{K}_{\Omega}$. Siano $U_1, U_2, \dots,$ le altre componenti connesse (contenute nella chiusura del disco) di $\mathbb{C}\setminus\widehat{K}_{\Omega}$.
    Se, per assurdo, $U_j \subset \Omega$, siccome $\partial U_j \subseteq \widehat{K}_{\Omega}$ allora $\overline{U}_j=U_j \cup \partial U_j \subset \subset \Omega$ contro il punto (vi). Supponiamo per assurdo che $\{U_j\}$ siano infinite. Per ogni $j \ge 1$ sia $z_j \in U_j \setminus \Omega$. A meno di sottosuccessioni $z_j \longrightarrow z_0 \in \mathbb{C} \setminus \Omega$.
    Sia $\rho>0$ t.c. $D(z_0, \rho) \cap \widehat{K}_{\Omega}=\emptyset$. Ma $D(z_0, \rho) \subset \mathbb{C}\setminus\widehat{K}_{\Omega}$ è connesso $\implies$ esiste $j_0$ t.c. $D(z_0, \rho) \subset U_{j_0}$ $\implies$ $U_j$ interseca $U_{j_0}$ per $j>>1$ $\implies$ $U_j=U_{j_0}$, assurdo.
  \end{nlist}
\end{proof}

\begin{thm}
  (Secondo teorema di Runge) Siano $\Omega_1 \subset \Omega_2 \subseteq \mathbb{C}$ aperti. Allora sono equivalenti:
  \begin{nlist}
    \item ogni $f \in \mathcal{O}(\Omega_1)$ può essere approssimata uniformemente sui compatti da funzioni in $\mathcal{O}(\Omega_2)$ [$(\Omega_1, \Omega_2)$ si dice \textit{coppia di Runge}];
    \item nessuna componente connessa di $\Omega_2\setminus\Omega_1$ è compatta;
    \item per ogni compatto $K \subset \Omega_1$ $\widehat{K}_{\Omega_2}=\widehat{K}_{\Omega_1}$;
    \item per ogni compatto $K \subset \Omega_1$ $\widehat{K}_{\Omega_2}\cap \Omega_1=\widehat{K}_{\Omega_1}$;
    \item per ogni compatto $K \subset \Omega_1$ $\widehat{K}_{\Omega_2}\cap \Omega_1$ è compatto.
  \end{nlist}
\end{thm}

\begin{proof}
  (iii) $\implies$ (iv) $\implies$ (v) è ovvio.

  (v) $\implies$ (i) Dato $K \subset \Omega_1$, poniamo $K'=\widehat{K}_{\Omega_2} \cap \Omega_1$ e $K''=\widehat{K}_{\Omega_2} \setminus \Omega_1$. $K' \cap K''=\emptyset, K \subseteq K'$, $K', K''$ sono compatti e $\widehat{K}_{\Omega_2}=K'\cup K''$.
  Sia ora $f \in \mathcal{O}(\Omega_1), \epsilon>0$. Applichiamo il primo teorema di Runge a $\widehat{K}_{\Omega_2}$ e $\tilde{f}$ t.c. $\tilde{f}\restrict{K'}\equiv f$ e $\tilde{f}\equiv 1$ in un intorno di $K''$ $\implies$ esiste $g \in \mathcal{O}(\Omega_2)$ t.c. $\|g-f\|_K \le \|g-f\|_{K'} \le \|g-f\|_{\widehat{K}_{\Omega_2}}<\epsilon$.

  (v) $\implies$ (iii) Dato $K \subset \Omega_1$, definiamo $K'$ e $K''$ come sopra. Applichiamo il primo teorema di Runge a $f\restrict{K} \equiv 0, f\restrict{K''}\equiv 1$.
  Se $z_0 \in K''$ esiste $F \in \mathcal{O}(\Omega_2)$ t.c. $\|F\|_K<1/2$ e $\|F-1\|_{K''}<1/2 \implies |F(z_0)|>1/2>\|F\|_K \implies z_0 \not\in \widehat{K}_{\Omega_2}$, assurdo $\implies K''=\emptyset \implies \widehat{K}_{\Omega_2} \subseteq \Omega_1 \implies \widehat{K}_{\Omega_2}=\widehat{K}_{\Omega_1}$.

  (i) $\implies$ (iv) È chiaro che $\widehat{K}_{\Omega_1} \subseteq \widehat{K}_{\Omega_2} \cap \Omega_1$. Sia $z_0 \in \Omega_1 \setminus \widehat{K}_{\Omega_1}$.
  Allora esistono $f \in \mathcal{O}(\Omega_1)$, $\epsilon>0$ t.c. $|f(z_0)|>\|f\|_K+\epsilon$. Sia $F \in \mathcal{O}(\Omega_2)$ t.c. $\|F-f\|_{K \cup \{z_0\}}<\epsilon/2$.
  Allora $|F(z_0)|>|f(z_0)|-\epsilon/2>\|f\|_K+\epsilon/2>\|F\|_K \implies z_0 \not\in \widehat{K}_{\Omega_2} \implies$ (iv).

  (ii) $\implies$ (iii) sia $U$ una componente connessa di $\Omega_2 \setminus K$ relativamente compatta in $\Omega_2$. Siccome $\partial U \subseteq K \subset \Omega_1$, $L=U \setminus \Omega_1$ è compatto in $\Omega_2$.
  Per assurdo, $a \in L$ e sia $C$ la componente connessa di $\Omega_2 \setminus \Omega_1$ che contiene $a$ $\implies$ $U \cup C$ è connesso; ma $U$ è una componente connessa di $\Omega_2 \setminus K \supset \Omega_2 \setminus \Omega_1 \implies U \supseteq C \implies C$ è relativamente compatto in $\Omega_2$ contro (ii), assurdo $\implies$ $L=\emptyset$ $\implies$ $U \subset \Omega_1$.
  Inoltre $\partial U \subseteq K$ $\implies$ $U$ è una componente connessa relativamente compatta in $\Omega_1$ $\implies$ $U \subseteq \widehat{K}_{\Omega_1}$ $\implies$ $\widehat{K}_{\Omega_2} \subseteq \widehat{K}_{\Omega_1}$ per il punto (vi) della proposizione \ref{inviluppo}. L'altra inclusione è ovvia.

  (iii) $\implies$ (ii) Sia $L$ la componente connessa di $\Omega_2 \setminus \Omega_1$ compatta in $\Omega_2$. Per un lemma topologico, possiamo trovare un intorno $U$ di $L$ relativamente compatto in $\Omega_2$ con $\partial U \subset \Omega_1$. Per il principio del massimo e per (iii), $\Omega_1 \supset \widehat{(\partial U)}_{\Omega_1}=\widehat{(\partial U)}_{\Omega_2} \supseteq U \supseteq L \implies L=\emptyset$.
\end{proof}

\begin{cor}
  Sia $\Omega \subset \mathbb{C}$ aperto. Ogni $f \in \mathcal{O}(\Omega)$ è approssimabile uniformemente sui compatti con polinomi $\iff$ $\mathbb{C}\setminus \Omega$ non ha componenti connesse compatte.
\end{cor}

\begin{proof}
  Prendiamo $\Omega_1=\Omega, \Omega_2=\mathbb{C}$.

  ($\Leftarrow$) Il secondo teorema di Runge ci dà l'approssimazione con funzioni in $\mathcal{O}(\mathbb{C})$, che a loro volta si approssimano con polinomi.

  ($\implies$) I polinomi sono funzioni in $\mathcal{O}(\mathbb{C})$, quindi si conclude sempre per il secondo teorema di Runge.
\end{proof}

\begin{thm}
  (Terzo teorema di Runge) Sia $\Omega \subset \mathbb{C}$ aperto, $\displaystyle \mathbb{C}\setminus \Omega=\bigcup_{\alpha}C_{\alpha}$ la decomposizione in componenti connesse. Sia $E \subset \mathbb{C}$ discreto con esattamente un punto in ciascuna $C_{\alpha}$ compatta. Allora ogni $f \in \mathcal{O}(\Omega)$ può essere approssimata con funzioni razionali con poli in $E$.
\end{thm}

\begin{proof}
  Sia $K \subset \subset \Omega$ compatto e sia $L=\widehat{K}_{\Omega}$. $\mathbb{C}\setminus L$ ha una componente connessa illimitata $U$ e un numero finito di componenti connesse limitate $W_1, \dots, W_p$.
  Inoltre per ogni $j$ $W_j \not\subset \Omega$, per cui interseca $\mathbb{C}\setminus\Omega$ e quindi contiene una componente connessa $C_{\alpha_j}$ di $\mathbb{C}\setminus \Omega$ $\implies$ $C_{\alpha_j}$ è compatta; sia $\{a_j\}=E\cap C_{\alpha_j}$.
  Sia $\Omega_0=\mathbb{C}\setminus\{a_1, \dots, a_p\} \supset L$. Le componenti connesse di $\Omega_0 \setminus L$ sono $U$ e $W_j \setminus \{a_j\}$. Nessuna di queste è relativamente compatta in $\Omega_0$.
  Per il secondo teorema di Runge, per ogni $f \in \mathcal{O}(\Omega)$ e per ogni $\epsilon>0$ esiste $F \in \mathcal{O}(\Omega_0)$ t.c. $\|F-f\|_L<\epsilon$. Sia $g_j$ la parte principale dello sviluppo di Laurent di $F$ in $a_j$ $\implies$ $F=h+g_1+\dots+g_p$ con $h=F-g_1-\dots-g_p \in \mathcal{O}(\mathbb{C})$.
  Sia $P \in \mathbb{C}[z]$ t.c. $\|P-h\|_L<\epsilon/(p+1)$. Se $\displaystyle g_j(z)=\sum_{n=-\infty}^{-1} c_n^{(j)}(z-a_j)^n$ possiamo trovare $N>0$ t.c. $\displaystyle \|g_j-\sum_{n=-N}^{-1} c_n^{(j)}(z-a_j)^n\|_L=\|g_j-g_{j,N}\|_L \le \epsilon/(p+1)$ per ogni $j=1, \dots, p$.
  Poniamo $G=P+g_{1,N}+\dots+g_{p,N}$. $G$ è razionale e $\displaystyle \|F-G\|_L \le \|h-P\|_L+\sum_{j=1}^p \|g_j-g_{j,N}\|_L<(p+1) \dfrac{\epsilon}{p+1}=\epsilon$.
\end{proof}
